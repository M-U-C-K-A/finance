% Options for packages loaded elsewhere
\PassOptionsToPackage{unicode}{hyperref}
\PassOptionsToPackage{hyphens}{url}
\documentclass[
]{article}
\usepackage{xcolor}
\usepackage[margin=1in]{geometry}
\usepackage{amsmath,amssymb}
\setcounter{secnumdepth}{-\maxdimen} % remove section numbering
\usepackage{iftex}
\ifPDFTeX
  \usepackage[T1]{fontenc}
  \usepackage[utf8]{inputenc}
  \usepackage{textcomp} % provide euro and other symbols
\else % if luatex or xetex
  \usepackage{unicode-math} % this also loads fontspec
  \defaultfontfeatures{Scale=MatchLowercase}
  \defaultfontfeatures[\rmfamily]{Ligatures=TeX,Scale=1}
\fi
\usepackage{lmodern}
\ifPDFTeX\else
  % xetex/luatex font selection
\fi
% Use upquote if available, for straight quotes in verbatim environments
\IfFileExists{upquote.sty}{\usepackage{upquote}}{}
\IfFileExists{microtype.sty}{% use microtype if available
  \usepackage[]{microtype}
  \UseMicrotypeSet[protrusion]{basicmath} % disable protrusion for tt fonts
}{}
\makeatletter
\@ifundefined{KOMAClassName}{% if non-KOMA class
  \IfFileExists{parskip.sty}{%
    \usepackage{parskip}
  }{% else
    \setlength{\parindent}{0pt}
    \setlength{\parskip}{6pt plus 2pt minus 1pt}}
}{% if KOMA class
  \KOMAoptions{parskip=half}}
\makeatother
\usepackage{color}
\usepackage{fancyvrb}
\newcommand{\VerbBar}{|}
\newcommand{\VERB}{\Verb[commandchars=\\\{\}]}
\DefineVerbatimEnvironment{Highlighting}{Verbatim}{commandchars=\\\{\}}
% Add ',fontsize=\small' for more characters per line
\usepackage{framed}
\definecolor{shadecolor}{RGB}{248,248,248}
\newenvironment{Shaded}{\begin{snugshade}}{\end{snugshade}}
\newcommand{\AlertTok}[1]{\textcolor[rgb]{0.94,0.16,0.16}{#1}}
\newcommand{\AnnotationTok}[1]{\textcolor[rgb]{0.56,0.35,0.01}{\textbf{\textit{#1}}}}
\newcommand{\AttributeTok}[1]{\textcolor[rgb]{0.13,0.29,0.53}{#1}}
\newcommand{\BaseNTok}[1]{\textcolor[rgb]{0.00,0.00,0.81}{#1}}
\newcommand{\BuiltInTok}[1]{#1}
\newcommand{\CharTok}[1]{\textcolor[rgb]{0.31,0.60,0.02}{#1}}
\newcommand{\CommentTok}[1]{\textcolor[rgb]{0.56,0.35,0.01}{\textit{#1}}}
\newcommand{\CommentVarTok}[1]{\textcolor[rgb]{0.56,0.35,0.01}{\textbf{\textit{#1}}}}
\newcommand{\ConstantTok}[1]{\textcolor[rgb]{0.56,0.35,0.01}{#1}}
\newcommand{\ControlFlowTok}[1]{\textcolor[rgb]{0.13,0.29,0.53}{\textbf{#1}}}
\newcommand{\DataTypeTok}[1]{\textcolor[rgb]{0.13,0.29,0.53}{#1}}
\newcommand{\DecValTok}[1]{\textcolor[rgb]{0.00,0.00,0.81}{#1}}
\newcommand{\DocumentationTok}[1]{\textcolor[rgb]{0.56,0.35,0.01}{\textbf{\textit{#1}}}}
\newcommand{\ErrorTok}[1]{\textcolor[rgb]{0.64,0.00,0.00}{\textbf{#1}}}
\newcommand{\ExtensionTok}[1]{#1}
\newcommand{\FloatTok}[1]{\textcolor[rgb]{0.00,0.00,0.81}{#1}}
\newcommand{\FunctionTok}[1]{\textcolor[rgb]{0.13,0.29,0.53}{\textbf{#1}}}
\newcommand{\ImportTok}[1]{#1}
\newcommand{\InformationTok}[1]{\textcolor[rgb]{0.56,0.35,0.01}{\textbf{\textit{#1}}}}
\newcommand{\KeywordTok}[1]{\textcolor[rgb]{0.13,0.29,0.53}{\textbf{#1}}}
\newcommand{\NormalTok}[1]{#1}
\newcommand{\OperatorTok}[1]{\textcolor[rgb]{0.81,0.36,0.00}{\textbf{#1}}}
\newcommand{\OtherTok}[1]{\textcolor[rgb]{0.56,0.35,0.01}{#1}}
\newcommand{\PreprocessorTok}[1]{\textcolor[rgb]{0.56,0.35,0.01}{\textit{#1}}}
\newcommand{\RegionMarkerTok}[1]{#1}
\newcommand{\SpecialCharTok}[1]{\textcolor[rgb]{0.81,0.36,0.00}{\textbf{#1}}}
\newcommand{\SpecialStringTok}[1]{\textcolor[rgb]{0.31,0.60,0.02}{#1}}
\newcommand{\StringTok}[1]{\textcolor[rgb]{0.31,0.60,0.02}{#1}}
\newcommand{\VariableTok}[1]{\textcolor[rgb]{0.00,0.00,0.00}{#1}}
\newcommand{\VerbatimStringTok}[1]{\textcolor[rgb]{0.31,0.60,0.02}{#1}}
\newcommand{\WarningTok}[1]{\textcolor[rgb]{0.56,0.35,0.01}{\textbf{\textit{#1}}}}
\usepackage{graphicx}
\makeatletter
\newsavebox\pandoc@box
\newcommand*\pandocbounded[1]{% scales image to fit in text height/width
  \sbox\pandoc@box{#1}%
  \Gscale@div\@tempa{\textheight}{\dimexpr\ht\pandoc@box+\dp\pandoc@box\relax}%
  \Gscale@div\@tempb{\linewidth}{\wd\pandoc@box}%
  \ifdim\@tempb\p@<\@tempa\p@\let\@tempa\@tempb\fi% select the smaller of both
  \ifdim\@tempa\p@<\p@\scalebox{\@tempa}{\usebox\pandoc@box}%
  \else\usebox{\pandoc@box}%
  \fi%
}
% Set default figure placement to htbp
\def\fps@figure{htbp}
\makeatother
\setlength{\emergencystretch}{3em} % prevent overfull lines
\providecommand{\tightlist}{%
  \setlength{\itemsep}{0pt}\setlength{\parskip}{0pt}}
\usepackage{booktabs}
\usepackage{longtable}
\usepackage{array}
\usepackage{multirow}
\usepackage{wrapfig}
\usepackage{float}
\usepackage{colortbl}
\usepackage{pdflscape}
\usepackage{tabu}
\usepackage{threeparttable}
\usepackage{threeparttablex}
\usepackage[normalem]{ulem}
\usepackage{makecell}
\usepackage{xcolor}
\usepackage{bookmark}
\IfFileExists{xurl.sty}{\usepackage{xurl}}{} % add URL line breaks if available
\urlstyle{same}
\hypersetup{
  pdftitle={Rapport Analytique des Cours Boursiers},
  pdfauthor={Système Automatisé HedgeFound},
  hidelinks,
  pdfcreator={LaTeX via pandoc}}

\title{Rapport Analytique des Cours Boursiers}
\usepackage{etoolbox}
\makeatletter
\providecommand{\subtitle}[1]{% add subtitle to \maketitle
  \apptocmd{\@title}{\par {\large #1 \par}}{}{}
}
\makeatother
\subtitle{AAPL \& GOOGL - Analyse Intraday}
\author{Système Automatisé HedgeFound}
\date{02 July 2025 à 01:04}

\begin{document}
\maketitle

\section{Analyse des Cours Boursiers}\label{analyse-des-cours-boursiers}

\subsection{Données Fondamentales}\label{donnuxe9es-fondamentales}

\begin{table}[!h]
\centering
\begin{tabular}{ccc}
\toprule
Ticker & PE & MarketCap\\
\midrule
\cellcolor{gray!10}{AAPL} & \cellcolor{gray!10}{32.37} & \cellcolor{gray!10}{-1.3 B\$}\\
GOOGL & 19.65 & 1.2 B\$\\
\bottomrule
\end{tabular}
\end{table}

\subsection{Performance Intraday}\label{performance-intraday}

\begin{table}
\centering
\resizebox{\ifdim\width>\linewidth\linewidth\else\width\fi}{!}{
\begin{tabular}{lrrrrrrrNA>{}NA}
\toprule
Ticker & Open & Close & High & Low & Max Variation & Min Variation & Performance\\
\midrule
\cellcolor[HTML]{28a745}{\textcolor{white}{\textbf{\cellcolor{gray!10}{AAPL}}}} & \cellcolor[HTML]{28a745}{\textcolor{white}{\textbf{\cellcolor{gray!10}{207.04}}}} & \cellcolor[HTML]{28a745}{\textcolor{white}{\textbf{\cellcolor{gray!10}{207.86}}}} & \cellcolor[HTML]{28a745}{\textcolor{white}{\textbf{\cellcolor{gray!10}{209.91}}}} & \cellcolor[HTML]{28a745}{\textcolor{white}{\textbf{\cellcolor{gray!10}{207.04}}}} & \cellcolor[HTML]{28a745}{\textcolor{white}{\textbf{\cellcolor{gray!10}{0.60}}}} & \cellcolor[HTML]{28a745}{\textcolor{white}{\textbf{\cellcolor{gray!10}{-0.40}}}} & \cellcolor[HTML]{28a745}{\textcolor{white}{\textbf{\cellcolor{gray!10}{0.40}}}} & NA & \textbf{NA}\\
\cellcolor[HTML]{dc3545}{\textcolor{white}{\textbf{GOOGL}}} & \cellcolor[HTML]{dc3545}{\textcolor{white}{\textbf{175.29}}} & \cellcolor[HTML]{dc3545}{\textcolor{white}{\textbf{175.87}}} & \cellcolor[HTML]{dc3545}{\textcolor{white}{\textbf{175.87}}} & \cellcolor[HTML]{dc3545}{\textcolor{white}{\textbf{173.58}}} & \cellcolor[HTML]{dc3545}{\textcolor{white}{\textbf{0.31}}} & \cellcolor[HTML]{dc3545}{\textcolor{white}{\textbf{-15.67}}} & \cellcolor[HTML]{dc3545}{\textcolor{white}{\textbf{0.33}}} & NA & \textbf{NA}\\
\bottomrule
\end{tabular}}
\end{table}

\subsection{Évolution des Cours}\label{uxe9volution-des-cours}

\pandocbounded{\includegraphics[keepaspectratio]{stock_report_files/figure-latex/price-evolution-1.pdf}}

\subsection{Analyse des Variations}\label{analyse-des-variations}

\pandocbounded{\includegraphics[keepaspectratio]{stock_report_files/figure-latex/variation-analysis-1.pdf}}

\subsection{Corrélation Entre
Actions}\label{corruxe9lation-entre-actions}

\begin{table}[!h]
\centering
\begin{tabular}{lrr}
\toprule
\multicolumn{2}{c}{Matrice de Corrélation} \\
\cmidrule(l{3pt}r{3pt}){1-2}
  & AAPL & GOOGL\\
\midrule
\cellcolor{gray!10}{AAPL} & \cellcolor{gray!10}{1.00} & \cellcolor{gray!10}{-0.17}\\
GOOGL & -0.17 & 1.00\\
\bottomrule
\end{tabular}
\end{table}

\subsection{Recommandations}\label{recommandations}

\begin{verbatim}
## \textbf{AAPL}: Tendance baissière - À surveiller
\end{verbatim}

\begin{verbatim}
## \textbf{GOOGL}: Tendance baissière - À surveiller
\end{verbatim}

\section{Annexes Techniques}\label{annexes-techniques}

\subsection{Données Brutes}\label{donnuxe9es-brutes}

\begin{table}
\centering
\resizebox{\ifdim\width>\linewidth\linewidth\else\width\fi}{!}{
\begin{tabular}{llrr}
\toprule
Ticker & Datetime & Price & Variation\\
\midrule
\cellcolor{gray!10}{AAPL} & \cellcolor{gray!10}{2025-07-01 13:30:00} & \cellcolor{gray!10}{207.04} & \cellcolor{gray!10}{NA}\\
AAPL & 2025-07-01 13:35:00 & 207.68 & 0.31\\
\cellcolor{gray!10}{AAPL} & \cellcolor{gray!10}{2025-07-01 13:40:00} & \cellcolor{gray!10}{207.53} & \cellcolor{gray!10}{-0.07}\\
AAPL & 2025-07-01 13:45:00 & 208.77 & 0.60\\
\cellcolor{gray!10}{AAPL} & \cellcolor{gray!10}{2025-07-01 13:50:00} & \cellcolor{gray!10}{208.78} & \cellcolor{gray!10}{0.00}\\
\addlinespace
AAPL & 2025-07-01 13:55:00 & 209.08 & 0.15\\
\cellcolor{gray!10}{AAPL} & \cellcolor{gray!10}{2025-07-01 14:00:00} & \cellcolor{gray!10}{208.97} & \cellcolor{gray!10}{-0.05}\\
AAPL & 2025-07-01 14:05:00 & 208.90 & -0.04\\
\cellcolor{gray!10}{AAPL} & \cellcolor{gray!10}{2025-07-01 14:10:00} & \cellcolor{gray!10}{209.66} & \cellcolor{gray!10}{0.37}\\
AAPL & 2025-07-01 14:15:00 & 209.59 & -0.03\\
\bottomrule
\end{tabular}}
\end{table}

\subsection{Métriques Statistiques}\label{muxe9triques-statistiques}

\begin{table}[!h]
\centering
\begin{tabular}{lrrr}
\toprule
Ticker & Volatility & Mean & Median\\
\midrule
\cellcolor{gray!10}{AAPL} & \cellcolor{gray!10}{0.6328} & \cellcolor{gray!10}{208.1438} & \cellcolor{gray!10}{208.01}\\
GOOGL & 0.5151 & 175.0479 & 175.07\\
\bottomrule
\end{tabular}
\end{table}

\begin{verbatim}

Pour utiliser ce rapport :

1. Installez les packages R requis :
```r
install.packages(c("ggplot2", "dplyr", "kableExtra", "scales", "quantmod", "lubridate", "gridExtra", "yfinance"))
\end{verbatim}

\begin{enumerate}
\def\labelenumi{\arabic{enumi}.}
\setcounter{enumi}{1}
\tightlist
\item
  Modifiez le fichier \texttt{run\_system.py} pour générer un PDF au
  lieu d'HTML :
\end{enumerate}

\begin{Shaded}
\begin{Highlighting}[]
\NormalTok{rmd\_result }\OperatorTok{=}\NormalTok{ subprocess.run([}\StringTok{"Rscript"}\NormalTok{, }\StringTok{"{-}e"}\NormalTok{, }\StringTok{"rmarkdown::render(\textquotesingle{}stock\_report.Rmd\textquotesingle{}, output\_format=\textquotesingle{}pdf\_document\textquotesingle{})"}\NormalTok{], ...)}
\end{Highlighting}
\end{Shaded}

Ce rapport amélioré inclut : - Une présentation professionnelle en PDF -
Des analyses fondamentales (PE, Market Cap) - Des graphiques enrichis
avec points clés - Des indicateurs de performance et volatilité - Une
analyse de corrélation - Des recommandations automatiques - Des annexes
techniques détaillées

\end{document}
